% Covariance Estimation notes

\documentclass[12pt, leqno]{article}
\usepackage{amsfonts, amsmath, amssymb}
\usepackage{mathtools}
\usepackage{fancyhdr}
\usepackage{hyperref}
\usepackage{graphicx}
%\usepackage{float}
%\usepackage{babel}
\providecommand{\abs}[1]{\lvert#1\rvert}
\providecommand{\norm}[1]{\lVert#1\rVert}
\newcommand{\macheps}{\epsilon_{\mbox{\scriptsize mach}}}
\let\oldhat\hat
\renewcommand{\vec}[1]{\mathbf{#1}}
\renewcommand{\hat}[1]{\oldhat{\mathbf{#1}}}
\def\rp{\ensuremath \mathbb{R}^p}
\def\rpp{\ensuremath \mathbb{R}^{p \times p}}
\def\s{\ensuremath\Sigma}
\def\om{\ensuremath\Omega}
\def\pd{\ensuremath\mathbb{P}^+}
\def\pg{\ensuremath\mathbb{P}_{{G}}}
\def\E{\ensuremath\mathbb{E}}
\def\normdist[#1]#2{\ensuremath \sim \mathcal{N} (#1,#2) }
\def\ndist1{\ensuremath \sim \mathcal{N}  (\mu, \sigma)}
\def\ndistvec{\ensuremath \sim \mathcal{N}_p (\vec{\mu}, \vec{\Sigma})}
\def\lra{\ensuremath\Leftrightarrow}
\newcommand\ind{\protect\mathpalette{\protect\independenT}{\perp}}
\def\independenT#1#2{\mathrel{\rlap{$#1#2$}\mkern2mu{#1#2}}}
\makeatletter
\newcommand{\distas}[1]{\mathbin{\overset{#1}{\kern\z@\sim}}}%
\newsavebox{\mybox}\newsavebox{\mysim}
\newcommand{\dist}[1]{%
  \savebox{\mybox}{\hbox{\kern3pt$\scriptstyle#1$\kern3pt}}%
  \savebox{\mysim}{\hbox{$\sim$}}%
  \mathbin{\overset{#1}{\kern\z@\resizebox{\wd\mybox}{\ht\mysim}{$\sim$}}}%
}
\makeatother

\begin{document}
\pagestyle{fancy}
\lhead{Syed Rahman}
\rhead{High dimensional data}

\begin{center}
{\large {\bf Homework 1}} \\
\end{center}

\paragraph{2.} Newton's method to minimize 
\[
f(X) = \norm{X}^3, X = \begin{pmatrix}
X_1\\ 
X_2\\ 
\end{pmatrix} \in\mathbb{R}^2.
\]
using a constant step size.

Thus, 
\[
\nabla f = 3\norm{X}X.
\]
and
\begin{align*}
\nabla^2 f  &=  3\begin{pmatrix}
\frac{X_1^2}{\norm{X}}+\norm{X}&\frac{X_1X_2}{\norm{X}}\\ 
\frac{X_1X_2}{\norm{X}}&\frac{X_2^2}{\norm{X}}+\norm{X}\\ 
\end{pmatrix}.
\end{align*}
As a result, 
\begin{align*}
(\nabla^2 f)^{-1}  &=  \frac{1}{3({\frac{X_1X_2}{\norm{X}}}^2 + X_1^2
                     + X_2^2 + \norm{X}^2) - {\frac{X_1X_2}{\norm{X}}}^2}\begin{pmatrix}
\frac{X_2^2}{\norm{X}}+\norm{X}&-\frac{X_1X_2}{\norm{X}}\\ 
-\frac{X_1X_2}{\norm{X}}&\frac{X_1^2}{\norm{X}}+\norm{X}\\ 
\end{pmatrix}.
\end{align*}
which implies,
\begin{align*}
  (\nabla^2 f)^{-1} (\nabla f) =& \frac{1}{6 \norm{X}^2}
\begin{pmatrix}
\frac{X_2^2}{\norm{X}}+\norm{X}&-\frac{X_1X_2}{\norm{X}}\\ 
-\frac{X_1X_2}{\norm{X}}&\frac{X_1^2}{\norm{X}}+\norm{X}\\ 
\end{pmatrix} \times 3\norm{X} \begin{pmatrix}
X_1\\ 
X_2\\ 
\end{pmatrix}
\\
=&
\frac{1}{2 \norm{X}}
\begin{pmatrix}
\frac{X_1X_2^2}{\norm{X}}+\norm{X}X_1-\frac{X_1X_2^2}{\norm{X}}\\ 
-\frac{X_1^2X_2}{\norm{X}}+\frac{X_1^2X_2}{\norm{X}}+\norm{X}X_2\\ 
\end{pmatrix} 
\\
=& \frac{1}{2} \begin{pmatrix}
X_1\\ 
X_2\\ 
\end{pmatrix} = \frac{X}{2}.
\end{align*}

Thus, Newton's method for this problem implies 
\begin{align*}
X^{k+1} =& X^{k} - \alpha   (\nabla^2 f)^{-1} (\nabla f) \\
=& X^{k}( 1 - \frac{\alpha}{2})  \\
=& X^{k-1}( 1 - \frac{\alpha}{2})^2  \\
\vdots \\ 
=& X^{0}( 1 - \frac{\alpha}{2})^k .
\end{align*}
As $k \to \infty,X^{k} \to 0$ for $0<\abs{( 1 - \frac{\alpha}{2})}<1 \iff 0<\alpha<4$.
\end{document}

