% HW2 high dimensional data

\documentclass[12pt, leqno]{article}
\usepackage{amsfonts, amsmath, amssymb}
\usepackage{amsthm}
\usepackage{mathtools}
\usepackage{fancyhdr}
\usepackage{hyperref}
\usepackage{graphicx}
\usepackage{caption}
\usepackage{subcaption}
\usepackage{float}
\usepackage{mathrsfs}
\usepackage{array} 
\usepackage{rotating}
%\usepackage{babel}
\providecommand{\abs}[1]{\lvert#1\rvert}
\providecommand{\norm}[1]{\lVert#1\rVert}
\newcommand{\macheps}{\epsilon_{\mbox{\scriptsize mach}}}
\let\oldhat\hat
\renewcommand{\vec}[1]{\mathbf{#1}}
\renewcommand{\hat}[1]{\oldhat{{#1}}}
\def\rp{\ensuremath \mathbb{R}^p}
\def\rpp{\ensuremath \mathbb{R}^{p \times p}}
\def\s{\ensuremath\Sigma}
\def\om{\ensuremath\Omega}
\def\pd{\ensuremath\mathbb{P}^+}
\def\pg{\ensuremath\mathbb{P}_{{G}}}
\def\E{\ensuremath\mathbb{E}}
\def\normdist[#1]#2{\ensuremath \sim \mathcal{N} (#1,#2) }
\def\ndist1{\ensuremath \sim \mathcal{N}  (\mu, \sigma)}
\def\ndistvec{\ensuremath \sim \mathcal{N}_p ( {\mu},  {\Sigma})}
\def\lra{\ensuremath\Leftrightarrow}
\def\stackrel#1#2{\mathrel{\mathop{#2}\limits^{#1}}}
\newcommand\ind{\protect\mathpalette{\protect\independenT}{\perp}}
\def\independenT#1#2{\mathrel{\rlap{$#1#2$}\mkern2mu{#1#2}}}
\makeatletter
\newtheorem{thm}{Theorem}[]
\newtheorem{lemma}{Lemma}[]
\newtheorem{defn}[thm]{Definition}
\newcommand{\sign}{\mathrm{sign}}
\newcommand{\distas}[1]{\mathbin{\overset{#1}{\kern\z@\sim}}}%
\newsavebox{\mybox}\newsavebox{\mysim}
\newcommand{\dist}[1]{%
  \savebox{\mybox}{\hbox{\kern3pt$\scriptstyle#1$\kern3pt}}%
  \savebox{\mysim}{\hbox{$\sim$}}%
  \mathbin{\overset{#1}{\kern\z@\resizebox{\wd\mybox}{\ht\mysim}{$\sim$}}}%
}
\makeatother

\begin{document}
\pagestyle{fancy}
\lhead{Syed Rahman}
\rhead{STA7934}

\begin{center}
{\large {\bf Homework 2 - Analysis of High Dimensional Data}} \\
\end{center}

\paragraph{} Let $f_i: \mathbb{R}^n \to \mathbb{R}$ and $P_i$ be
bounded polyhedral subsets of $\mathbb{R}^n$ with non-empty intersection. We want to solve
the following problem:
\begin{align*}
\textbf{PRIMAL:}&
\min \sum_{i=1}^m f_i(x), \quad x \in \mathbb{R}^n\\
&\text{subject to } x \in P_i, i = 1,...,m
\end{align*}
Note that the constraint $x \in P_i \iff A_ix \leq b_i, A_i \in
\mathbb{R}^{n_i \times n}, b_i \in \mathbb{R}^{n_i}$ and that these
are all convex. Thus we can
reformulate the above problem as:
\begin{align*}
\textbf{PRIMAL:}&
\min \sum_{i=1}^m f_i(x) \\
&\text{subject to } A_ix \leq b_i, \quad i = 1,...,m
\end{align*}
Note that so long as $f_i(x) > -\infty \quad \forall x$ that are
feasible, we can always find a maximum, say $p^*$. In addition, if the $f_i$ are convex, then $\sum_{i=1}^m f_i(x)$ is
convex and this is a convex problem, which means that local minima are
also global minima. This assumption is also need to apply KKT 1-4 and
Slater's condition to show that strong duality holds as we will later see.
\paragraph{} The Lagrangian is equal to 
\begin{align*}
L(x,\lambda_1,...,\lambda_m) = \sum_{i=1}^m f_i(x) + \sum_{i=1}^m
  \lambda_i' (A_ix - b_i), \quad \lambda_i \in \mathbb{R}^n
\end{align*}
Thus the dual is 
\begin{align*}
g(\lambda_1,...,\lambda_m) &= \inf_{x} L(x,\lambda_1,...,\lambda_m) \\
&= \inf_{x} \sum_{i=1}^m f_i(x) + \sum_{i=1}^m
  \lambda_i' (A_ix - b_i)
\end{align*}
Once way to minimize $L(x,\lambda_1,...,\lambda_m)$ is to take
derivative with respect to $x$. If $f_i$ is differentiable, 
\begin{align*}
\nabla_x
L(x,\lambda_1,...,\lambda_m) &= \nabla_x \sum_{i=1}^m f_i(x) + \sum_{i=1}^m
  \lambda_i' (A_ix - b_i) \\
&= \sum_{i=1}^m \nabla_x f_i(x) + \sum_{i=1}^m A_i' \lambda_i
\end{align*}
Thus let $x^*$ be such that $ \sum_{i=1}^m \nabla_x f_i(x^*) +
\sum_{i=1}^m A_i' \lambda_i = 0$. This also implies
that KKT-4 holds. If $f$ is not differentiable then
we have to repeat the same argument using subdifferentials and show
that $0$ is in the set of subdifferentials. It is clear that KKT-1
holds as $A_ix - b_i \leq 0 \quad \forall i=1,...,m$. 
\paragraph{}The dual problem can be formulated as:
\begin{align*}
\textbf{DUAL:}&
\max g(\lambda_1,...,\lambda_m) \\
&\text{subject to } \lambda_i \succeq 0, \quad i = 1,...,m
\end{align*}
Note that the dual problem is always concave and hence a local
maximum is a global maximum, although there may be more than one
global maxima. Let's call this $d^*$. 
From the formulation, it is clear that KKT-2 holds as well. Finally, we need to show
that KKT-3, or complementary slackness, holds. For this we consider 2
situations - one where $x \in (\cap_{i=1}^m P_i)^o$ and the other where $x
\in \partial (\cap_{i=1}^m P_i )$. Here $A^o$ denotes the interior of
A and
$\partial(A)$ denotes the boundary of A.
If $x \in (\cap_{i=1}^m P_i)^o$ and $f_i$ is
convex for $i = 1,...,m$ we can apply Slater's condition
directly. Thus strong duality holds, i.e. the duality gap is 0 and
there is no need to apply KKT conditions. In the other case when $x
\in \partial (\cap_{i=1}^m P_i )$, suppose we have that $A_ix = b_i, ~  i =
1,...,k$ and $A_ix \leq b_i, ~  i =
k+1,...,m$. One way to ensure complementary slackness holds is to set
$\lambda_i = 0 \in \mathbb{R}^n, ~ i = k+1,...,m$, which
makes these into non-binding constraints. Recall that to show that there is
no duality gap in
this case we need KKT 1-4 in addition to $f_i$ being convex and
differentiable. These are all satisfied if we pick feasible
$(x,\{\lambda_i\}_{i=1}^m)$ according to the primal and dual problems
(KKT 1-2) such that the following additional
conditions hold: 
\begin{enumerate}
\item $\sum_{i=1}^m \nabla_x f_i(x) +
\sum_{i=1}^m A_i' \lambda_i {=} 0$. (KKT-4)
\item $\lambda_i = 0 \in \mathbb{R}^n$ whenever $A_ix \leq b_i$. (KKT-3)
\end{enumerate}
 
\end{document}